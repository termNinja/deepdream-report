% !TEX encoding = UTF-8 Unicode

\documentclass[a4paper]{article}

\usepackage{color}
\usepackage{xcolor}
\usepackage{url}
\usepackage[T2A]{fontenc} % enable Cyrillic fonts
\usepackage[utf8]{inputenc} % make weird characters work
\usepackage{graphicx}
\usepackage{amsthm}
\usepackage{amsmath}
\usepackage{booktabs,hhline}

\usepackage{multirow}       % used for making multirow tables (Nemanja)

%\usepackage[english,serbian]{babel}
\usepackage[english,serbianc]{babel} %ukljuciti babel sa ovim opcijama, umesto gornjim, ukoliko se koristi cirilica

\usepackage[unicode]{hyperref}
\hypersetup{colorlinks,citecolor=green,filecolor=green,linkcolor=blue,urlcolor=blue}

\newtheorem{primer}{Пример}[section] %ćirilični primer
%\newtheorem{primer}{Primer}[section]

\newtheorem{definic}{Дефиниција}
\newcommand{\norm}[1]{\left\lVert#1\right\rVert}

\begin{document}

%TODO smisliti naslov
\title{Google DeepDream \\ \small{Семинарски рад у оквиру курса\\Истраживање података\\ Математички факултет}}

\author{Немања Мићовић\\ nmicovic@outlook.com}
\date{}
\maketitle

\abstract{
    Dodati posle.
}

\tableofcontents

\newpage

% ---------------------------------------------------------------------------------------------------------------------
\section{Увод}
% ---------------------------------------------------------------------------------------------------------------------
Неуронске мреже данас постижу изузетне резултате на многим пољима вештачке интелигенције.
Нананаанана увод Нананаанана увод Нананаанана увод
Нананаанана увод Нананаанана увод Нананаанана увод
Нананаанана увод Нананаанана увод Нананаанана увод
Нананаанана увод Нананаанана увод Нананаанана увод
Нананаанана увод Нананаанана увод Нананаанана увод
Нананаанана увод Нананаанана увод Нананаанана увод

% ---------------------------------------------------------------------------------------------------------------------
\section{Неуронске мреже}
% ---------------------------------------------------------------------------------------------------------------------

% ---------------------------------------------------------------------------------------------------------------------
\subsection{Перцептрон}
% ---------------------------------------------------------------------------------------------------------------------
% ---------------------------------------------------------------------------------------------------------------------
\subsection{Конволутивне неуронске мреже}
% ---------------------------------------------------------------------------------------------------------------------

% ---------------------------------------------------------------------------------------------------------------------
\section{Google Deep dream}
% ---------------------------------------------------------------------------------------------------------------------

% ---------------------------------------------------------------------------------------------------------------------
\section{Закључак}
% ---------------------------------------------------------------------------------------------------------------------
Zakljucak...

\addcontentsline{toc}{section}{Literatura}
\appendix
\bibliography{literature}
\bibliographystyle{plain}


\end{document}
